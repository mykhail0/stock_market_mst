\documentclass{article}
\usepackage[polish]{babel}
\usepackage[utf8]{inputenc}
\usepackage[T1]{fontenc}

\title{Generowanie minimalnego drzewa rozpinającego dla podanych spółek}
\author{
    Adam Boguszewski \and
    Mykhailo Shevchenko \and
    Wiktor Zdrojewski
}
\date{21 czerwca 2020}


\begin{document}

\maketitle

\section{Streszczenie}
Stworzyliśmy programik, który potrafi przetworzyć arkusze kalkulacyjne
na minimalne drzewo rozpinające. Pojedyńczy arkusz ma zawierać notowania
spółki wraz z datami, w których te notowania zostały pobrane.

\section{Jak korzystać z programu}
Program ma całkiem prosty interfejs i na razie nie oferuje dużo możliwości
dla użytkownika. Program dostaje 3 argumenty:

Jako pierwszy argument dostaje ścieżkę do folderu, gdzie
znajdują się arkusze kalkulacyjne z danami o firmach, jedna firma per arkusz.

Kolejne dwa argumenty są w postaci YYYYMMDD i oznaczają przedział czasowy,
z którego będą rozważane notowania.

Przykładowe wywołanie:\newline
./program ../data/gielda/wse\_stocks 20200000 20200621

W wyniku obliczeń program wyświetla minimalne drzewo rozpinające spółek.

\section{Źródło danych}
Działanie programu jest zależne od postaci danych w arkuszu kalkulacyjnym.
Na stan teraźniejszy program jest w stanie obsłużyć jedynie jedno źródło danych
(natomiast łatwo dodać obsługę danych z innych źródeł, tworząc odpowiednią funkcję parsującą plik).
Dane można pobrać z danej strony\newline
https://stooq.pl/db/h/?fbclid=IwAR1qQ\_xn5ryHV-McmCWIVlk-tnhERoyKog9u-O3Zw4QjDqO5mxRtTNF2K5U\newline
Należy pobrać archiwum i potem go rozpakować. Archiwum może być dowolne, byle by z kolumny Dzienne ASCII.

\section{Opis działania programu}
Program pobiera ceny zamknięcia z każdego dnia z każdego arkusza z podanej ścieżki.
Zatem tworzy tablicę spółek, w której są przechowywane te wartości, po czym
ucina tablicę zgodnie z podanym przedziałem czasowym.

Następnie program usuwa z tablicy firmy, które
nie mają podanych notowań dla dużej liczby dni.
Program dla każdej spółki rozważa stosunek dni bez notowań do
liczby dni w przedziale czasowym i jeśli ten stosunek przekracza pewien próg,
ustalony wewnątrz programu, to dana spółka nie jest rozważana w dalszym działaniu programu.
Aktualnie próg jest ustalony na $0\%$, tzn. wszystkie firmy, które mają chociaż jeden dzień
nie odnotowany, są usuwane.

W kolejnym kroku program oblicza wektor dla każdej spółki na podstawie cen zamknięcia.
Wektorem jest względna zmiana ceny, czyli stopa zwrotu:
$$ R(t) = \frac{Z(t+dt)-Z(t)}{Z(t)},$$
gdzie $Z(t)$ to cena zamknięcią w dniu $t$, a $dt$ w naszym przypadku będzie równe $1$.

Potem obliczane są współczynniki korelacji i odległości zgodnie ze wzorem
$$ d(X, Y) = \sqrt{2(1 - p(X, Y))}, $$
gdzie $p(X,Y)$ to współczynnik korelacji Pearsona pomiędzy spółkami $X$ i $Y$.
Na podstawie tej tablicy w następnym kroku wykonywania programu powstaje
ważona klika, gdzie wierzchołki to nazwy spółek, a krawędź pomiędzy $X$ i $Y$
ma wagę równą $d(X, Y)$. Ta klika jest przekształcana algorytmem Kruskala
i dostajemy drzewo minimalnej rozpiętości.

Projekt został napisany w Python'ie.
Do przetwarzania tablic i arkuszów została użyta biblioteka \textsl{pandas}.
Do tworzenia, przetwarzania i wypisywania grafów została użyta biblioteka \textsl{graph tool}.


\end{document}
